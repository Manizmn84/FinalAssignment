\documentclass[12pt]{article}

\usepackage[utf8]{inputenc}
\usepackage{geometry}
\geometry{a4paper, margin=1in}
\usepackage{graphicx}
\usepackage{hyperref}
\usepackage{fancyhdr}

\setlength{\headheight}{15pt}
\pagestyle{fancy}
\fancyhf{}
\rhead{Computer Workshop Course}
\lhead{Final Assignment}
\rfoot{Page \thepage}

\title{Final Assignment}
\author{Mani Zamani}
\date{January 2025}

\begin{document}

\maketitle
\newpage
\tableofcontents
\newpage

\section{repository link}
\paragraph{\url{https://github.com/Manizmn84/FinalAssignment}}


\section{Git and GitHub}
\subsection{Repository Initialization and Commits}

\paragraph{To set up a repository, first of all,we must have a GitHub account. After creating a GitHub account or logging into the account, we perform the following steps:}

\begin{itemize}
    \item Go to the repository section in your account
    \item Click on New Repository
    \item Enter the explation in the description field
    \item Choose wrepository is public or private
    \item choose add README
    \item click on create repository 
\end{itemize}
\subsection{GitHub Actions for LaTeX Compilation}
\paragraph{Provide a walkthrough of setting up GitHub Actions to automatically compile your LaTeX
document and any challenges you encountered?}
\paragraph{To launch a compiler for LaTech with Github action, you must set up a workflow and enter your codes for compilation in the main.yml file.\\ It didn't work with tags on my system, so I had to go with push.
One of the challenges I faced was that when I wanted to do it with tags, it wouldn't work and I had to proceed with a push-up mode.}

\section{Exploration Tasks}
\subsection{Vim Advanced Features}
\paragraph{Explore and document 3 advanced features of Vim that were not covered in class?}
\paragraph{hear of three advance feature in vim:}

\begin{enumerate}
    \item One of Vim's features is macros, with the help of which you can record a sequence of keystrokes that can be executed with only one keystroke. The reason for using this feature is clear from the fact that in text editing and in programming, some tasks need to be executed over and over again. We can call this task repetitive tasks, for example, we can refer to formatting code or generating boilerplate text.
    \item Vim's another feature is a tabbed interface which allows you to open many files and switch between them easily .this can be usefull to save your time while you working. Tabs support multiple windows, making it easy to work with multiple files in same time.
    \item last feature is the folding, which is provided for programmers and some people who work with huge documents, this feature allows you to fold and unfold sections of text, making it easy to navigate and work with large files.this feature is useful for saving time and no need to spend time to transport in large documents.
\end{enumerate}

\subsection{Memory profiling}
\subsubsection{Memory Leak}
\paragraph{In short, explain what memory leaks are and how they might happen in your program?}
\paragraph{A memory leak occurs when a programmer allocates memory and does not free it}
\paragraph{If we don't free the allocated memory, the possibility of memory leak is high}


\subsubsection{Memory profilers}
\paragraph{Read about a tool called Valgrind and write about their purpose and how it helps when memory leaks happen.?}
\paragraph{valgrind is amazing tool witch is use for identification allocated memory.if we allocate a part of memory and we did not free it in purpose and then we open this application we will see that it has tell us about the allocated memory and free memory and the size of memory witch is allocated}


\end{document}